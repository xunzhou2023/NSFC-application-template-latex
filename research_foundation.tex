{\color{MsBlue} \subsection{\sihao \kaishu \quad \ (二)研究基础与工作条件 }}


{\sihao \color{MsBlue} \kaishu 1.{\bfseries 研究基础}(与本项目相关的研究工作积累和已取得的研究工作成绩);}

本课题的主要研究人员长期致力于时空数据挖掘、分析、智能技术的研究工作,尤其在时空事件分析挖掘、时空图数据挖掘、时空数据生成式学习等具体领域具有较为丰富的研究经验,对国内外研究进展和现状有全面的了解把握,相关研究成果已发表在重要的国际会议和期刊中,包括IEEE Transactions on Knowledge and Data Engienering (TKDE),International Journal on Geographical Information Sciences (IJGIS)(地理信息系统领域顶级期刊),KDD, NeurIPS,AAAI,ICDE,WWW,ICLR等,并多次获得最佳/优秀论文奖。

项目负责人周逊目前为哈尔滨工业大学(深圳)三级教授、博士生导师,曾任美国爱荷华大学商业分析系(信息系统方向)终身副教授,系博士项目主任,并双聘于爱荷华大学应用与计算数学和信息学两个系任博士生导师。长期致力于时空数据挖掘、时空大数据分析、时空计算等领域的研究,攻读博士学位期间师从国际知名时空数据挖掘和时空数据库专家,明尼苏达大学Shashi Shekhar教授(IEEE和AAAS Fellow)。2016年获得美国国家科学基金NSF CRII奖(被名校广泛认为是计算机领域的pre-CAREER奖)。目前为IEEE高级会员,并入选国家优青(海外)人才项目。在数据挖掘、机器学习和时空计算等相关领域的会议和期刊上发表论文100余篇,包括议KDD,AAAI,NeurIPS,WWW,CIKM,ICDM,SDM,TKDE,KAIS,Geoinformatica等CCF-A/B类会议和期刊40余篇,其他JCR Q1期刊20余篇,获得包括ICDM'21 Best Paper Award, SDM'19 Best Applied Data Science Paper在内的5次国际国内一流会议最佳或优秀论文奖。联合主编Springer出版的地理信息系统大百科全书一部。在美期间承担了美国国家科学基金NSF、美国交通部、爱荷华大学等机构资助的7项与时空大数据分析、挖掘与智能相关的研究项目。

针对本项目的研究内容,申请人及研究团队成员具有较为丰富的相关研究经验和扎实的研究基础,详述如下:

在时空数据的生成式学习方面,申请人先后提出了基于生成对抗网络GAN的栅格化交通流量数据生成学习方法TrafficGAN~\cite{zhang2019trafficgan-x},CurbGAN~\cite{zhang2020curb-x},Mest-GAN~\cite{zhang2022mest-x},$C^3$-GAN~\cite{zhang2021c-x}等,并提出了基于条件生成学习方法的疫情期间人流量数据生成模型COVID-GAN~\cite{}与STORM-GAN~\cite{}。这些成果均发表于KDD、ICDM、KAIS、SIGSPATIAL、ACM TIST等CCF-A/B类会议、重要时空计算国际会议和JCR Q1类期刊上。其中STORM-GAN入选2022年ICDM(CCF-B类)会议最佳论文候选。所指导的美国爱荷华大学博士生获得学校优秀博士毕业论文资助奖金。%并入选学校Dare to Discover活动作为优秀科研人才被报道。
上述成果使得申请人对时空数据的生成式学习研究有了充足的技术准备,本课题将基于以上研究成果,对时空图结构表示的时空数据上的生成式学习进行深入研究。

在时空事件的分析、挖掘与建模上,申请人团队针对时空聚集事件~\cite{}、时空扩散事件~\cite{}、交通事故与犯罪等点过程事件~\cite{}等多种类时空事件进行了深入研究,提出了对其动态时空足迹进行刻画和挖掘的多种方法。相关工作发表在AAAI, TKDE,SIGSPATIAL等重要会议和期刊上。这些前期成果为本项目中对时空事件进行双层次建模和刻画的任务提供了良好铺垫。

针对时空数据分析中的自相关性和异质性的挑战,申请人提出了一系列基于重要性采样~\cite{}、空间划分~\cite{}、空间集成学习~\cite{}、知识迁移~\cite{}等技术提高预测准确性的方法,并将研究内容应用到交通事故和犯罪事件预测等方面,取得了良好的效果,相关研究获得美国交通部和NVidia公司项目资助。其中一篇研究异质性数据预测的论文获得CCF-B类ICDM'21年度唯一最佳论文奖~\cite{}。这些前期的研究积累为本课题中处理子图生成中的时空自相关性和异质性挑战提供了重要的技术储备和扎实的理论基础。

在时空图数据的分析和挖掘上,申请人提出了基于轨迹数据的时空图可达性分析查询算法~\cite{}。申请人还提出了基于时空有向图挖掘的城市聚集事件监测和预报方法~\cite{}。以上工作均发表在CCF-A类期刊IEEE TKDE上。此外,申请人还提出了基于时空图卷积神经网络的交通事故预测和分析技术~\cite{}和基于图神经网络的视频中物体跟踪分析方法~\cite{}。这些前期研究使团队具备了充分的时空图数据挖掘分析和深度学习的经验和方法积累。

以下为课题主要研究人员已发表的相关论文:
\bibliographystyle{ieeetrNSFC}
\bibliography{mywork}
%申请人用\LaTeX 写过几篇文章,包括自己的博士论文。

{\sihao \color{MsBlue} \kaishu 2.{\bfseries 工作条件}(包括已具备的实验条件,尚缺少的实验条件和拟解决的途径,包括利用国家实验室、国家重点实验室和部门重点实验室等研究基地的计划与落实情况);}

本项目依托单位为哈尔滨工业大学(深圳)。哈尔滨工业大学是我国的知名高等学
院,国家“双一流”重点建设的大学。科研工作和人才培养紧紧围绕国防建设和国民经
济建设,与一批国内知名企业和一些相关的国内外研究机构有稳定的联合、合作与协作
关系。哈尔滨工业大学(深圳)计算机科学与技术学院设有计算机科学与技术国家重点
一级学科,具有学士、硕士、博士学位授予权和博士后流动站。计算机科学与技术学科
在2012 年教育部学科评估中位居全国第四,2017 年被评为“双一流”建设学科和全国第四轮学科评估A 类学科,2018 年入选广东省高水平大学重点建设学科。

申请人所在的哈工大深圳“智能与计算研究院”由特聘校长助理、国家杰青获得者张民教授领导,2023年获批成立深圳市重点实验室,目前有7名国家级青年人才,与海外高水平研究机构和院校建立了长期的学术和交流。课题组已有GPU 服务器集群一套(96 个GPU),另有其它独立CPU/GPU 服务器20余台,能充分保障本项目研究开展的有序进行。

{\sihao \color{MsBlue} \kaishu 3.{\bfseries 正在承担的与本项目相关的科研项目情况}(申请人和主要参与者正在承担的与本项目相关的科研项目情况,包括国家自然科学基金的项目和国家其他科技计划项目,要注明项目的资助机构、项目类别、批准号、项目名称、获资助金额、起止年月、与本项目的关系及负责的内容等);}

无。

{\sihao \color{MsBlue} \kaishu 4.{\bfseries 完成国家自然科学基金项目情况}(对申请人负责的前一个已资助期满的科学基金项目(项目名称及批准号)完成情况、后续研究进展及与本申请项目的关系加以详细说明。另附该项目的研究工作总结摘要(限500字)和相关成果详细目录)。}

无。
%不告诉你。

{\color{MsBlue} \subsection{\sihao \kaishu \quad \ (三)其他需要说明的情况 }}

{\sihao \color{MsBlue} \kaishu 1. 申请人同年申请不同类型的国家自然科学基金项目情况(列明同年申请的其他项目的项目类型、项目名称信息,并说明与本项目之间的区别与联系)。 }

无。

{\sihao \color{MsBlue} \kaishu 2. 具有高级专业技术职务(职称)的申请人或者主要参与者是否存在同年申请或者参与申请国家自然科学基金项目的单位不一致的情况;如存在上述情况,列明所涉及人员的姓名,申请或参与申请的其他项目的项目类型、项目名称、单位名称、上述人员在该项目中是申请人还是参与者,并说明单位不一致原因。}

无。

{\sihao \color{MsBlue} \kaishu 3. 具有高级专业技术职务(职称)的申请人或者主要参与者是否存在与正在承担的国家自然科学基金项目的单位不一致的情况;如存在上述情况,列明所涉及人员的姓名,正在承担项目的批准号、项目类型、项目名称、单位名称、起止年月,并说明单位不一致原因。}

无。

{\sihao \color{MsBlue} \kaishu 4. 其他。}

无。