\subsection{1.2 国内外研究现状及发展动态分析}
%2023-01-20: 根据2023年面上项目申请书正文的官方MsWord模板,对本模板的字号和少量蓝色文字做了更新。
对于本课题研究的``基于自相关子图生成的事件条件下时空图数据推演技术'',目前国内外现有的研究成果中还缺乏直接有效的方法。 与本课题直接相关的研究工作主要包括三个方面:(1)时空属性图预测方法,(2)基于深度学习的图数据生成方法,以及(3)其他时空数据的生成学习方法,包括非深度学习的图生成模型和非图结构的时空数据生成学习方法。下面对这三类现有研究现状进行分析和归纳,并总结出现有研究的不足,从而引出本课题计划研究的创新性内容及其必要性。

\subsubsection{1.2.1 时空属性图预测方法}
时空属性图上的数据预测问题近年来一直是研究热点。其典型应用场景包括交通流量预测、出行需求预测、环境观测值预测、传染病防控等~\cite{jin2023spatio}。给定一个时空图序列$G_t=(\mathcal{V}, \mathcal{E}_t, \mathbf{X}_t, \mathbf{E}_t)$,$t\in [T-L+1, T]$其中$\mathcal{E}_t$、$\mathbf{X_t}$和$\mathbf{E_t}$分别为图节点邻接关系、节点属性和边属性值在$t$时刻的值,$L$为历史数据窗口长度,时空图数据预测的目的是通过模型拟合出未来$l$个时间段内$\mathbf{X}$以及$\mathbf{E}$的值。非概率的时空图预测模型直接根据输入对输出值进行预测,而非获取输出变量的概率分布\footnote{我们将概率预测模型归入生成模型在下一章节讨论。}。故而,时空图数据预测的结果不具有随机性和变化性。一旦模型训练完成,则针对每个输入的输出结果则完全确定,且不具有生成新数据的能力。下面针对近年来有代表性的相关工作进行简要综述和分析。

Yu等人提出了时空图卷积网络模型(STGCN)对交通路网上的车流量进行预测~\cite{yu2018spatio}。该工作将车流量的观测站点作为图节点,车流量观测值定义为节点上的属性序列,将站点相邻关系定义为边,并以站点间距离为边权重,通过图上所有节点的历史观测数据预测未来的车流量。论文中提出了针对空间维度信息提取的谱图卷积操作以及针对时间维度信息提取的时间门卷积,并将两者连接成时空卷积操作模块ST-Conv以更好地融合所提取的时空信息。Guo等人对相同的问题提出了层次化图卷积的方法~\cite{Guo19attention},将输入的图结构聚类成不同区域,针对宏观的区域连接图和微观的路网进行分别卷积,同时加入了图注意力机制来提高结果的准确度。Chen等人提出了残差循环图卷积网络(Res-RGN)~\cite{chen2019gated}。该模型基于扩散图卷积(diffusion convolution)和GRU来对时空图信息进行提取,并通过对网络里的残差连接增加控制门来更精准地控制信息流动,从而增加预测的准确性。Zhao等人提出了基于空间图卷积(Graph Convolution)与门控循环单元(GRU)结合的时空图预测方法T-GCN~\cite{zhao2019t}。该方法通过显式刻画时空自相关性来进行交通流量预测并获得了较之前方法更好的准确度。

针对时空自相关性如何刻画的问题,相关工作进一步提出了一系列方法。例如,Diao等人提出了基于图结构拉普拉斯矩阵估计的方法来动态地学习不同图节点之间的数据相关性。通过张量分解的方法将实时交通流量数据分解为全局流量和本地流量,并分别估计两个部分的拉普拉斯矩阵,进一步提高了预测的精度~\cite{diao2019dynamic}。Bai等人~\cite{bai2020adaptive}同样提出动态地对时空相关性矩阵进行估计。作者直接学习节点嵌入向量$E$,并用$EE^T$对时空图的拉普拉斯矩阵进行预测。最后,作者将节点嵌入学习模块、空间邻接矩阵预测模块嵌入到GRU结构里,组成了自适应的图卷积循环网络(AGCRN),对时空属性图上的交通流量时间序列进行多步预测。Geng等人利用时空图卷积网络对网约车需求量进行预测~\cite{geng2019spatiotemporal}。特别地,该论文从多个层面刻画时空图节点的相似性,包括城市功能区相似性、路网连通性和空间距离等,并将这些相似性度量定义为多图结构,通过多频道图卷积操作、时间卷积操作等模块组合成最终的模型ST-MGCN。Ye等人~\cite{ye2021coupled}提出了一个多层次的图时空卷积网络,针对神经网络不同的节点层分别生成不同的邻接矩阵,再通过对所有图卷积层的聚集生成最终的结果。

Dai等人在路网实时信息基础上添加了用户搜索的导航信息来提高交通流量预测的准确度,提出了H-STGCN模型~\cite{dai2020hybrid}并构建包含路段相关性的复合邻接矩阵来提高预测精度。Li等人指出了相关方法在描绘时空图节点间复杂时空相关性上的不足,并提出建构多视图的节点间相关性时序图来解决这一问题。论文作者继而提出了时空融合图神经网络(STFGNN)对所生成的时序图进行融合学习以进一步提高交通流量预测的精准度~\cite{li2021spatial}。Zheng等人提出了一种基于图编码器-解码器架构的交通流量预测方法GMAN~\cite{zheng2020gman}。该方法将多头注意力机制应用到时空数据当中,分别对时间序列的嵌入和节点空间信息的嵌入学习注意力权重,再添加控制门进行融合,从而更精准地、动态地捕捉数据中的时空自相关性。Zhang等人提出了一种时空多尺度层次化的学习方法~\cite{zhang2021traffic},利用自监督注意力机制学习方法提取不同时间和空间尺度单位(例如城市区域)的表征嵌入。同时,作者设计了一个图扩散卷积网络来学习区域之间的相关性信息,进而对未来各个区域的交通流量进行预测。

除了常见的图卷积神经网络及其变形外,部分研究提出了基于常微分方程等物理引导的预测模型。例如,Choi等人~\cite{choi2022graph}采用基于神经网络的可控偏微分方程(NCDE)来进行预测。具体地,作者定义每个图节点属性随时间变化的路径可以表示为一个连续可微的函数$z(t)$,其起始值为$z(0)$。对于某一时刻$T$的序列值$z(T)$,可以用$\frac{z(t)}{dt}$对$t$在$[0,T]$上求积分的方式获得。该方法通过训练一个时间维度的神经网络$f_\theta(t)$来拟合每个节点上的时序微分$\frac{z(t)}{dt}$,再用一个空间维度的神经网络$g$将节点信息融合,最终得到未来属性的预测值。Ji等人~\cite{ji2022stden}将路网中的交通流量预测问题看成一个势能场(potential energy field)驱动的物理模型,车流量被定义为相邻图节点之间的势能梯度。作者将势能$z$在时间上的微分拟合为一个图卷积残差神经网络,通过求解常微分方程来对该网络进行训练并最终预测出交通流量分布。

最近,相关研究开始关注图数据预测中的时空非静态性或异质性挑战。Jiang等人针对时空图数据,提出了基于图元学习的预测方法~\cite{jiang2023spatio}。通过对训练数据中的节点学习一个嵌入模式库,将具有代表性的嵌入表征储存下来。针对每个需要计算嵌入的测试节点,算法可以从模式库中提取与当前节点最为接近的一组表征进行组合,从而得到测试节点的嵌入向量并进行进一步的解码。该方法可以提高模型应对时空异质性的能力。Cini等人提出了一种新的训练节点嵌入的方法~\cite{cini2023taming},通过训练一个全局分量和一个本地嵌入分量来提高节点嵌入的本地化表达能力,进而帮助模型针对不同节点进行具有针对性的预测。Zhou等人~\cite{zhou2023maintaining}针对时空图预测问题中的分布外泛化问题(Out-of-Distribution)进行了研究,并提出了对时间和空间不变关系的提取方法,使得在数据分布变化的情况下预测准确度提高。

综上,此类方法的预测方式是直接数值预测,而非概率学习,故而\textbf{不具有随机性,也无法生成新数据}。另外,
现有的时空图数据的预测方法多数基于时间上的连续性这一基本假设,\textbf{更适合于对规律性的、非突发性的时空数据进行预测}。尚无方法讨论针对假设性事件条件的推演问题。

\subsubsection{1.2.2 基于深度学习的图数据生成方法}
图生成模型近年来成为另一个研究热点。其中一个主要方面是学习如何生成符合训练数据分布的图结构~\cite{guo2022systematic,zhu2022survey,li2018learning}。图结构生成在化学和生物结构发现方面有广泛应用。在生成图拓扑结构的同时,部分方法也可以生成节点和边上的属性信息。形式化地该问题可定义如下~\cite{zhu2022survey}:定义图结构为${G}$ = $\mathcal{(V, E, \mathbf{X}, \mathbf{E})}$, 其中$\mathcal{V}$为所有$N$个图节点集合,$\mathcal{E} \subseteq \mathcal{V}\times \mathcal{V}$ 为边集合,%$\mathcal{A_G}$为图$\mathcal{G}$的邻接矩阵。
$\mathbf{X}$ %\in \mathbb{R}^{{N}\times D}$
为节点上的维属性值,$\mathbf{E}%\in \mathbb{R}^{{N}\times N\times F}
$为边属性集。当节点、边上的属性具有时空特性,例如随时间变化时,则为时空图。给定由$M$个符合上述定义的图样本组成的训练数据集$\mathcal{G} = \{G_i\}_{i=1}^M$,图生成学习的目的是学习一个$\mathcal{G}$服从的概率分布$p(\mathcal{G})$,并通过从中进行采样$G_{new}\sim p(\mathcal{G})$来生成新的图结构。当训练数据集$\mathcal{G}$中的样本不包含标签信息时,此类学习任务称为无条件生成。当$\mathcal{G}$中样本$G_i$同时含有相应的类别标签$c_i\in C$或其他输入信息时,学习的目标变为获得条件概率分布$G_{new}\sim p(\mathcal{G}|C)$,即条件生成。

图生成学习的主要思路是训练一个参数化的模型,将图样本$G$映射到一个低维的隐变量$z$上。假设隐变量的概率分布为$p(z)$,通过对该变量的采样获取不同$z$的样本并通过解码过程生成出相对应的图结构和节点及边上的相关属性值。在生成技术方面,常见的深度学习模型包括可变分自编码器VAE,对抗生成网络GAN,标准化流(normalized flow)、扩散去噪模型(diffusion models)及其变种等方法。根据生成方式可分为一次性生成(one-shot generation)和顺序生成(sequential generation)。以下部分将图生成方法分为三类:无条件生成、有条件生成、时空图生成,并针对图生成具有代表性的技术进行详细论述。

\textbf{(a)无条件图数据生成}。\underline{一次性生成方法}通过利用对噪声隐变量进行解码生成出新的邻接矩阵和节点/边上的属性值。例如Anand等人利用深度对抗生成网络结构对蛋白质结构图的邻接矩阵进行生成~\cite{anand2018generative}。Li等人利用图反卷积神经网络~\cite{li2021deconvolutional},通过设计谱卷积和小波去噪等方法生成新的图结构。Fan等人同样利用GAN模型及其变形来生成图数据的邻接矩阵和节点属性~\cite{fan2019labeled}。Ma等人提出了正则化的图变分自编码器(VAE)模型来生成合理有意义的图结构以保证原子价和连通性的约束条件~\cite{ma2018constrained}。Niu等人提出了一种排列无关的图生成模型EDP-GNN~\cite{niu2020permutation}。该模型的输入包括一个节点数固定的邻接矩阵,并利用基于分数的去噪模型(score-based diffusion model)来生成图节点和边上的数据。Guo等人提出了基于变分自编码器的节点-边解耦的方法NED-VAE来分别学习控制节点、边和节点-边共同属性的三种不同的隐变量分布,并通过对其分别解码的方式来生成边和节点上的属性~\cite{guo2020interpretable}。Vignac等人~\cite{vignac2021top}提出了非独立随机同分布假设的采样方法,使得生成的图序列避免过度相似或产生大量不合实际的图结构。Luo等人提出了基于标准化流的深度学习方法GraphDF,即学习一个可逆的图编码器,对图结构进行编码和解码。GraphDF将离散的隐变量值映射到图节点和边属性上,并通过这种方法生成更加真实合理的分子结构图~\cite{luo2021graphdf}。以上方法均属于一次性图生成方法。此类方法需要指定生成图的最大节点数,故而能一次性生成的图规模有限。

\underline{顺序生成方法}则按照一定的顺序,对每个节点、边或子图结构进行先后生成。此类方法的优点在于其生成模式更为灵活,可以生成出大小不同的图结构。例如,%此类工作和链接预测(link prediction)相类似。
You等人提出了一种经典的图生成模型GraphRNN~\cite{you2018graphrnn}。该模型将图生成任务转化成每个节点的邻接矩阵行向量的生成问题,利用RNN模型进行自回归式的生成。每一步生成时,模型根据已经生成的图结构来推断下一节点的邻接关系向量的分布,并通过采样生成出相应的边。Zhang等人提出了无环有向图的生成模型D-VAE~\cite{zhang2019d}。该模型在生成时根据已有的图结构隐式状态变量,对下一个节点的类别进行采样并生成节点。随后,再对该节点和已有图结构之间的边进行逐次生成来完成全图生成。此类方法每次生成一个节点及其相关的边。在生成较大的图结构时,存在计算效率低的问题。

Liao等人提出了一种基于图循环注意力网络的高效图拓扑生成方法~\cite{liao2019efficient}。该方法每次生成的并不是一个节点,而是若干节点组成的节点块(block)。每次基于已有的图结构,该方法生成一个包含多个节点的节点块,并利用基于信息传递的图注意力网络模型来更新每个节点的表征向量。之后,根据目前所有节点块的表征向量来计算出新节点块对应的邻接矩阵行的内容。这种方法相较于每次生成一个节点的方法大幅提高了效率。

Jin等人提出一种基于层次化图案生成的方法来进行分子结构图生成~\cite{jin2020hierarchical}。具体地,该方法将一个图结构分解为多个图案(motif)和连接他们的关节点。通过对图案、关节点分别的概率分布学习来刻画整个图结构的概率分布。生成时,模型采用自回归的模式,分别预测新的图案、新图案和已生成图的连接点属性、以及新图案与已生成图连接的位置。通过不断将新的图案添加进已有的图来完成顺序生成。

%(4)生成条$\mathbf{C} \in \mathbb{R}^{|\mathcal{V}|\times|T|\times d_C}$,
%目前较为常见的图生成模型根据其生成目标,可分为:(1)节点生成、(2)边生成,(3)边和节点联合生成。在节点生成方面,

\textbf{(b)有条件图数据生成。} 
相比于无条件生成,图数据的条件生成方法基本采取类似的思路,但是需要额外考虑生成条件的表示和嵌入问题。根据输入中生成条件的类型,可以分为基于离散标签和嵌入向量的生成以及基于图结构条件的图生成。以下详细介绍各类相关工作以及其独特之处。 

\underline{基于条件向量的条件图生成:} Simonovsky等人提出了经典模型GraphVAE~\cite{simonovsky2018graphvae}。在论文中,作者指出该模型可以用来进行条件生成,在训练时将样本的离散标签$y$与输入图的嵌入进行连接送入编码器,生成隐变量$z$的概率分布,并将标签$y$与采样后的$z$拼接再送入解码器进行图生成。论文在实验中使用了原子比例直方图等等低维标签$y$来进行相应类别的分子结构图生成。

Yang等人提出了条件图生成模型CONDGAN~\cite{yang2019conditional}。该模型基于可变对抗生成网络架构(variational generative adversarial networks),针对每一个图节点学习一个嵌入表征$z_i$,然后将生成条件标签向量$C$与$z_i$拼接得到新的节点嵌入表征$\hat{z_i}$。通过对节点嵌入$\hat{z_i}$和$\hat{z_j}$的解码函数$f$的学习,模型得到$i$与$j$之间边的概率分布并通过对其进行采样来最终生成图结构:$\hat{A_{i,j}} = Sigmoid(f(\hat{z_i})^Tf(\hat{z_j}))$。

Vignac等人提出了基于去噪扩散模型的DiGress方法~\cite{vignac2022digress},对具有离散的节点属性值和离散边属性值的属性图进行生成。特别地,作者指出该方法也可以用来进行条件式生成。通过训练一个回归预测模块$g_\eta(G^t) = \hat{y}$,利用图$G$的加噪版本$G^t$对生成条件$y_G$进行预测,模型可以在逆降噪的过程中对无条件生成模型进行引导,将每一步降噪的概率分布更新为$G^{t-1}\sim p_\theta(G^{t-1}|G^t)p_\eta(\hat{y}|G^{t-1})$,其中$p_\theta$和$p_\eta$为训练得到的无条件生成和条件预测模型。用这种方法,DiGress模型可以针对生成条件$y_G$进行相应的数据生成。
%
%综上,如果输入的生成条件可以表示为一个离散标签或者嵌入向量,通常可以将其与其他输入进行拼接然后传递给模型。具体地,条件向量可以在训练的不同阶段添加到模型的不同位置,包括编码器、解码器,或者采样阶段。\textbf{但是,目前研究中尚无针对具有时空属性的事件作为生成条件的研究}。
%\underline{基于序列条件的图数据生成:}

\underline{基于源图结构的条件生成:} 此类方法中,输入的生成条件$C$可表示为一个图结构。此类问题在分子结构生成和社交关系推断中较为常见。具体地,给定一个源图$G_S(\mathcal{V_S}, \mathcal{E_S}, F_S, E_S)$,生成与之对应的目标图$G_T(\mathcal{V_T}, \mathcal{E_T}, F_T, E_T)$。根据源图和目标图的关系,相关技术又分为两类。当$\mathcal{V_S} = \mathcal{V_T}$,$F_S = F_T$时,源图和目标图具有相同的节点,需要生成的是边的邻接关系和属性。针对此类问题,相关研究集中在如何学习源图中每个节点和每条边的嵌入信息并据此生成目标图中的边信息。

当$\mathcal{V_S} \ne \mathcal{V_T}$时,则需要重新生成一个具有类似图属性但节点和源图不同的目标图。部分方法通过对源图进行表征学习得到含有整体图结构信息的图嵌入向量,并用这个向量作为生成条件生成出目标图。另一部分方法则直接在源图上进行``编辑'',例如添加或删除节点、边等,最终得到目标图。下面简单总结具有代表性的方法。

Guo等人提出了基于图生成条件的图生成方法(又称图-图翻译)GT-GAN~\cite{guo2022deep}。该方法基于对抗生成网络,将生成问题中的生成条件表示为一个有向图$G_X$,训练一个图生成器$T$使其根据$G_X$和给定的一个噪声数据$U$来生成目标图数据$G_{Y'}$。同时,该模型训练一个判别器$D$来判别生成的图数据是否为真实数据。整个模型训练采用对抗生成模式,通过优化目标函数$\mathcal{L}(T,D) = \mathbb{E}_{G_X, G_Y}[\log D(G_Y|G_X)]+\mathbb{E}_{G_X,U}[\log (1-D(T(G_X,U)|G_X))]$来学习参数化的生成器$T$和判别器$D$。为了学习条件图的嵌入信息,作者提出了有向图卷积操作,即对每个节点的上游邻居和下游邻居分别进行图卷积操作并将结果求和。作者同时还提出了图反卷积操作,即定义对上下游邻居节点的两个不同的图卷积和函数。通过将这两个和函数分别应用在每个结点的嵌入向量上来生成其对上下游邻居节点的影响值。

Lim等人将深度图生成网络应用在分子结构生成这一具体任务中,并提出了基于骨架结构的分子结构预测方法~\cite{lim2020scaffold}。该方法使用VAE作为深度学习架构,通过给定的分子骨架结构作为条件,来生成分子结构的其他部分。Maziarka等人提出了一种基于Cycle-GAN的化合物结构生成模型Mol-CycleGAN~\cite{maziarka2020mol}。该模型将输入的化合物结构作为生成条件,生成出结构类似并且在特性上更优化的化合物结构。Lee等人提出了一种化学分子结构生成的新方法MOOD~\cite{lee2023exploring}。该方法使用基于分数的去噪模型(score-based diffusion model),通过引入一个新的超参数来刻画生成数据在分布外(out-of-distribution)泛化的能力。针对对某一分子特征条件,MOOD可以生成具有该属性的小概率图结构,避免生成结果和训练集过于相似的问题。

%\textbf{需要指出的是},以上方法中所有的图节点或边上均不包含空间属性,也没有针对空间属性设计相关的生成技术,可以称为\textbf{非时空图生成}。以下部分介绍目前提出的针对时空属性图进行生成的相关方法。

\underline{时空图数据深度生成}方面的现有工作较少,主要研究集中在如何生成带有时空信息(例如坐标、时间戳)的图结构和节点属性。例如,
Guo等人提出利用图变分自编码器(Graph VAE)来解耦数据中的空间、拓扑、空间-拓扑三种信息的隐变量,并分别学习不同的VAE结构来对三类输入的噪声采样信号进行解码。最后用联合生成的方式同时生成出新的时空图,包括节点的时空属性和图的邻接关系~\cite{du2022disentangled}。Du等人将上述模型拓展到时空领域,用同样的思路生成了节点具有时空信息的图序列~\cite{du2022disentangled}。Zhang等人提出了对动态时序图的生成方法~\cite{zhang2021disentangled}。相比于之前工作,该方法可以解耦并分别学习时间相关和图信息相关的隐变量分布,从而生成拓扑结构随时间变化的图结构。Zhou等人提出了一种基于时序随机游走(Temporal Random Walk)的动态时序网络生成方法。该方法首先通过时序随机游走对图样本进行采样并学习这些样本序列的分布。在生成图结构时,首先生成序列集合并将其组合来生成完整的空间动态图。

综上,相关工作均以单个图或图序列作为生成样本,并且假设样本的隐变量之间是独立随机同分布的。\textbf{目前尚无图生成技术考虑空间上自相关的子图生成方法}。同时,在图数据条件生成方面,\textbf{现有工作也没有关于利用具有几何形状的时空事件作为生成条件的研究}。
%通过学习一个GAN模型将条件概率$p_\theta(G_Y|G_Y)$

\iffalse
\subsubsection{1.2.3 图嵌入和子图采样技术}
子图采样方法主要目的是帮助提高图神经网络的训练效率。在大规模图数据上,用全图数据进行训练在计算上不可行。如何提高训练效率并尽可能地将有代表性的信息纳入训练集,是子图采样策略研究的核心问题。现有的采样方法的对象是图嵌入和表示学习。例如,通过随机行走(random walk)的方法对结点的关联边进行采样,以涵盖邻居节点的信息。然而,这些图采样方法都不是以子图数据生成为目标。子图数据生成需要决定将哪些子图划分为生成训练的样本。同时,这些子图也将在生成阶段作为生成单元。其核心目标是划分出数据分布相似,学习难度较低的子图。\fi



%Zhang等人对动态时空图的预测方法进行了研究~\cite{zhang2022dynamic}。在动态图中,图的边集合$\mathcal{E}$和邻接矩阵是随时间动态变化的。该问题的一个难点是如何处理分布漂移,即图数据服从的分布随着时间有所变化。为了解决这一问题,论文作者提出了对时空信号进行解耦,并

\subsubsection{1.2.3 其他时空数据生成学习模型}
时空数据的生成学习有较长的研究历史。传统的\textbf{非深度学习生成模型}对点过成、栅格数据和图数据进行过研究,并提出了一系列方法。在时间点过程(temporal point process)上的生成模型包括泊松分布~\cite{pasupathy2010generating}、Hawkes过程~\cite{gonzalez2016spatio}、标值点过程(marked point process)~\cite{cressie2015statistics},Cox过程~\cite{brix2001spatiotemporal}等。这些模型又被扩展到时空维度上。例如自激发点过成(Self-exciting point process)是Hawkes过程在时空维度上的扩展~\cite{mohler2011self},假设已发生的事件在时间和空间上对未来事件的发生概率产生影响,事件概率分布通过求解背景强度函数(intensity function)和已发生事件影响的核函数(kernel function)来描述。在传统栅格数据或图结构数据上的生成模型还包括马尔可夫随机场及其变种,包括高斯马尔科夫随机场、马尔可夫链等。马尔可夫随机场用无向图结构来表示一个联合概率分布,图的节点表示随机变量,边表示随机变量之间的依赖关系。当变量之间满足马尔可夫属性,即任意系节点上变量的概率分布只与其邻居节点变量相关,而与其他节点上变量无关时,称该联合分布为马尔可夫随机场(Markov Random Field, MRF)~\cite{li1994markov}。当图结构为链式结构时,称为马尔可夫链~\cite{ching2006markov}。当每个变量的分布函数为高斯分布时,称高斯马尔可夫随机场(GMRF)。马尔可夫链在自然语言处理等领域有大量的应用,马尔可夫随机场模型在图像分割、去噪等处理技术上发挥了重要作用。马尔可夫模型可以为数据提供自相关性的统计学特性,是描述空间数据的重要基本模型。然而,由于自身不具备复杂特征提取的机制,MRF等模型通常只用来对简单的隐变量空间进行建模。

在深度学习广泛应用之后,时空数据的深度生成学习近年来受到更多关注~\cite{gao2022generative}。研究者将深度学习模型与上述生成方法结合并提出了\textbf{基于深度学习的时空数据生成学习}方法。例如,\underline{针对点过程数据},Zhang等人和Zuo等人分别提出基于Transformer和自注意力机制的Hawkes点随机过程生成模型~\cite{zuo2020transformer,zhang2020self}。Okawa等人~\cite{okawa2019deep}提出了基于多源信息和深度学习的时空self-exciting模型来预测未来事件。Yuan等人~\cite{yuan2023spatio}利用去噪扩散概率模型(Denoising Diffusion Probablistic Model)学习如何生成未来点过程的时空分布信息。另一部分研究者将在自然语言处理中取得良好表现的序列生成模型应用到\underline{轨迹数据的生成}任务中。例如,Wang等人提出了基于LSTM模型的轨迹生成方法进行数据增强~\cite{wang2022deep}。Feng等人提出了基于对抗生成模型和Transformer的轨迹生成模型~\cite{feng2020learning}。Yuan等人提出了基于时空生成对抗模仿学习的方法生成用户的移动行为数据~\cite{yuan2022activity}。

另外一大类工作是基于深度学习的\underline{栅格数据生成}。栅格数据是对时空数据进行网格化划分并对每个时空单元进行聚集之后得到的统计量(例如总交通流量,平均车速,犯罪数量等)。此类数据和图片或视频数据的结构类似。相应地,主要研究集中在利用VAE或GAN等经典模型进行数据生成。一部分工作将聚集后的时空数据看成是图片或者图片序列,并利用对抗生成网络(GAN)或其变种进行数据生成。申请人也在这方面进行了研究并发表了一系列论文。例如Zhang等人提出了条件生成对抗网络(Conditional GAN)来进行时空数据生成,包括Curb-GAN~\cite{zhang2020curb}, TrafficGAN~\cite{zhang2019trafficgan}等方法,将生成区域表示为二维矩阵,将生成条件和生成数据拼接后送入GAN网络进行对抗性训练~\cite{zhang2021c,zhang2022mest,zhang2022strans}。Bao等人利用条件生成网络应用到疫情下城市出行模式的估计上,将疫情控制政策作为生成条件来生成相应的人口流动数据~\cite{bao2022covid,bao2020covid}。这些方法在原有的cGAN模型上增强了时空自相关性的捕捉,例如利用动态卷积(dynamic convolution)和时间上的多头注意力机制、空间统计嵌入等技术。


%2023-01-29\&02-01: 根据几位老师的建议,对section的缩进,“参考文献”四个字的大小、字体和居左等做了调整。官方模板中阿拉伯数字不加粗,因此也做了相应的调整。
以上时空数据深度生成方法主要集中在非图结构上,无法直接应用到子图生成问题中。基于传统非深度学习的模型,例如马尔可夫随机场,无法对复杂数据进行直接信息提取。\texttbf{目前尚无将二者结合,对基于马尔可夫随机场隐变量的图数据进行深度生成学习的技术}。

%时空生成模型通常假设训练集的高维数据$\mathcal{D}$采样自低维嵌入空间上的同一概率分布(例如高斯、均匀分布等)$z\sim p(z)$。这个假设在时空数据上存在不合理性。首先,时空数据的分布服从地理学第一定理和第二定律。这导致临近地区的数据及其低维嵌入空间上的映射具有天然的近似性,即$p(z)$在空间上并非独立分布。例如,在城市交通数据生成场景中,面积占5\%以下的城市核心商业区的平均访问量可能达到面积占20以上的城郊区域访问量的100倍以上。而将这两类区域混合在训练数据中进行训练会导致模型所学习到的数据分布趋向于整个区域的平均值,无法准确生成出概率小、数值极端的核心商业区的。

\subsubsection{1.2.4 总结分析}
综上,与本课题相关的方法可归纳为如下三类:

\textbf{(1)判别式时空图数据预测技术}:该类方法基于回归或分类模型,对每个时空节点或边上的属性值进行预测。其模型学习到的是从输入变量到输出变量的直接映射,而非目标变量服从的概率分布,故而预测结果缺乏变化和统计上的不确定性。同时,也无法对新数据进行概率生成,故而无法解决时空图条件推演的问题。

\textbf{(2)图数据的生成学习技术}:该类方法的基本思想是将每个图样本数据通过一个参数化的神经网络映射到一个隐变量的先验概率分布上(通常为标准高斯分布),通过对隐变量采样,模型可以生成出与之对应新的图结构或图数据分布。这类方法具有以下三个不足,使其无法解决本课题的时空图推演问题中:(i)\underline{生成方式问题}。时空图规模较大。现有的全图一次性生成模型难以训练且无必要。而对图节点的顺序生成则会积累误差,并破坏图节点之间的自相关性信息。更符合时空图推演需要的是局部生成的方式。(ii)\underline{生成条件问题}。现有生成模型通常基于结构简单的生成条件,例如离散化标签、序列表征向量、图结构等。尚无方法考虑如何将同时具有多维度几何特征和图拓扑特征的时空事件作为生成条件。(iii)\underline{模型假设问题}。现有图生成方法其根本假设是隐变量的独立随机性。而在时空图推演中,各局部子图之间存在自相关性,即不同的子图样本之间是自相关而非独立的。同时,各子图之间由于时空异质性的存在,也很可能不符合同分布假设。故而,将已有方法应用在子图生成上将造成数据缺乏真实性和统计上的合理性。

\textbf{(3)其他相关时空数据生成技术},例如非图结构上的时空数据生成方法没有对时空图结构进行显式刻画和建模,故而所学习的模型无法直接应用在图数据的生成上。而基于马尔可夫随机场的图结构模型缺乏直接提取和处理复杂特征的能力,故而也不能直接用来进行图数据的条件式推演。

%同时,这些方法的生成条件通常为\underline{简单的离散标签}(例如有或无疫情,司机ID等),或者用\underline{张量表示的连续属性值}(例如本地平均出行需求等)。这些方法不足以解决本项目提出的点随机过程或多边形向量数据表示的“复杂”条件下的数据生成问题。

基于以上分析,事件条件下的时空图数据推演是一个新问题,难以用已有技术解决。针对\underline{生成方式}这一根本问题,本课题提出新的策略:即对时空图进行子图划分,并对各子图上数据进行符合自相关性的生成。为实现这一策略,我们需要设计一套与现有方法不同的理论、模型和算法。因而,申请人提出针对\textbf{基于自相关子图生成的多尺度时空图条件推演技术}这一课题进行研究。具体来说,实现本课题的目标需要解决以下问题:

\textbf{(1)生成条件的表示问题}:时空事件作为生成条件,具有多维度的时空属性以及不同的内涵和影响。如何对时空事件进行统一建模和表征学习,使其能够和时空图数据进行融合,是一个基础问题。

\textbf{(2)自相关性子图数据的生成学习问题}:如何建立一个生成模型,使其能够对时空图的局部子图数据进行生成,并保证子图间的时空自相关性?由于目前主流的深度图生成模型都没有采用自相关性假设,本课题需要创新性地何设计一个将子图数据映射到自相关隐变量空间的生成模型,并建立相关的理论基础、模型和算法。

\textbf{(3)多尺度下的子图生成方法问题}:如何使基于自相关子图生成的推演模型可以满足图不同尺度的推演需求?在使用模型进行数据推演时,要生成的区域可能会有不同的大小。例如,对某街区的交通流量推演只需要小尺度的数据生成,而在城市群交通网络上进行人口移动的推演则需要进行大范围大尺度的数据生成。在这种情况下如何高效地完成数据生成是一个具有挑战性的问题。

\textbf{(4)子图生成模型的时空泛化性问题}:如何使基于自相关子图生成的推演模型在不同子图和区域上具有良好的空间泛化性?由于事件条件的时空稀疏性以及时空数据天然的异质性,一个全局模型可能在局部地区难以达到良好的生成效果。如何对模型的时空泛化性进行增强也是实现本课题目标的关键。
%本项目拟解决的问题与上述问题皆不相同。首先,时空属性图一般规模较大(例如多城市区域路网)。对整个图进行联合数据生成既造成样本稀疏、计算开销巨大等困难,也无实际必要。故而,本问题的生成对象是局部子图上的数据。其次,时空属性图生成的生成条件复杂,既包含同为图结构的历史数据,也包含随机点过程或多边形区域表示的事件。最后,时空子图上的数据生成问题的生成样本来源于同一个时空图结构的不同区域,根据空间统计学基本假设和地理学第一第二定理,各子图样本的分布既非独立也非同分布,违反常见生成模型使用的独立随机高斯分布的假设。

%基于以上分析,解决时空属性子图上的条件生成问题需要与上述几类方法不同的技术路线。具体地,进行时空属性图数据生成需要解决以下三个关键问题:(1)如何对复杂和高维时空生成条件进行合理高效的时空条件嵌入表示方法,(2)如何建立数学模型和优化目标,将子图空间映射到具有时空自相关性和异质性且方便采样的统计分布上。(3)如何高效地对深度学习模型进行训练以学习上述映射关系,使学习到的模型在不同子图上具有可迁移性。

%围绕着这三个核心问题,课题组将探索\emph{基于时空超图的马尔可夫随机场模型}、\textit{基于空间子图划分的采样策略}、\textit{基于时空元学习和原型子图(Prototypical subgraph)的生成模型}等三项关键技术,从而解决上述科学问题。

% \begin{figure}[!th]
% \begin{center}
% \includegraphics[width=2in]{fig-example.eps}
% \caption{{\kaishu 插图可以使用EPS、PNG、JPG等格式。}}
% \label{fig:example}
% \end{center}
% \end{figure}

%2023-02-06: 根据@Readon的commits,做了如下修改:1. 更正了一处蓝字部分:``(三)其他需要说明的问题'' $\rightarrow$ ``(三)其他需要说明的情况'' 。这个非常重要!2. 设置 AutoFakeBold=3,这样楷体粗体稍微细一点,和官方模板更加接近。3. 调整了页面空白的宽度,大家可以自行微调。作为\LaTeX 菜鸟,非常感谢Readon的专业建议!更多专业的修改请见\url{https://github.com/Readon/NSFC-application-template-latex}

%2023-12-04:一转眼2024年的申请又要开始了。主要做了两点修改:1. 把图的caption字体调整为楷体。2. 设置AutoFakeBold=2,个人感觉这样和MsWord模板中的粗体更接近一点。等官方模板更新之后我再做相应更新。
\iffalse

\vskip 2mm
%\subsubsection{1.1 声明}
%{\bfseries \color{red} 注意!!!非国家自然科学基金委官方模版!!!}由个人根据官方MsWord模版制作。本模版的作者尽力使本模版和官方模版生成的PDF文件视觉效果大致一样,然而,并不保证本模版有用,也不对使用本模版造成的任何直接或间接后果负责。 不得将本模版用于商用或获取经济利益。本模版可以自由修改以满足用户自己的需要。但是如果要传播本模版,则只能传播未经修改的版本。使用本模版意味着同意上述声明。

%祝大家基金申请顺利!如有问题,请发送邮件到 \href{mailto:ryanzz@foxmail.com}{ryanzz@foxmail.com}。中了基金的也欢迎反馈。$\smiley$

%\subsubsection{1.2 使用说明}\label{sss:instruction}

\begin{enumerate}
\item 编译环境:推荐使用跨平台编译器texlive2017以后的版本,编译顺序为:xelatex+bibtex+xelatex(x2)。windows用户可以用命令行运行批处理文件getpdf.bat,linux用户可以运行runpdf。
\item 本模版力求简单,语句自身说明问题(self explanatory)。几乎只需要修改本tex文件即可满足排版需求,没有sty cls 等文件。用户掌握最基本的\LaTeX 语句即可操作,其余的可以用搜索引擎较容易地获得。
\item 参考文献样式:对作者个数作了限制以适合申请书,当作者个数小于等于5个时,予以全部保留,当作者个数大于5个时,只保留前3个,加et al。参考文献需要放在bib文件中。样式由ieeetrNSFC.bst控制。
\end{enumerate}

%\subsubsection{1.3 图、公式和参考文献的引用示例}
尽管不大可能会用到像下面这样简单的公式:
\begin{equation}
\label{eq:ex}
\sqrt[15]{a}=\frac{1}{2},
\end{equation}
我们还是用公式(\ref{eq:ex})举个数学公式的例子。同时,我们也不大可能会用到一个长得很像\LaTeX 的图,但是还是引用一下图\ref{fig:example}。图\ref{fig:example}并没有告诉我们关于Jinkela\cite{John1997,Smith1900}的任何信息,也没有透露它的产地\cite{Piter1992}。尽管如此,最近的研究表明,Feizhou非常需要Jinkela\cite{John1997}。

\fi







%对作者个数作了限制以适合申请书
%当作者个数小于等于5个时,予以全部保留,当作者个数大于5个时,只保留3个,加et al
%\newpage
{\footnotesize
\bibliographystyle{ieeetrNSFC}
\bibliography{related}}
\newpage
